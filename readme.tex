\documentclass[english]{article}
\usepackage[T1]{fontenc}
\usepackage[latin9]{inputenc}
\usepackage{hyperref}
\usepackage{url}
\usepackage{babel}
\begin{document}

\title{README for ``On informational nudging and control of payoff-based learning'' MATLAB source code}


\author{Robin Guers, Cedric Langbort and Dan Work}
\maketitle
\begin{abstract}
This document describes the implementation of the simulation used in ``On informational nudging and control of payoff-based learning'' by R. Guers C. Langbort and D. Work, accepted for publication in 4th IFAC Workshop on Distributed Estimation and Control in Networked Systems, NecSys'13.
A preprint of the article is available for download on the third
author's website. The source code is hosted at \url{https://github.com/RobinGuers/On-Informational-Nudging}.
\end{abstract}

\section{License}

This software is licensed under the \emph{University of Illinois/NCSA
Open Source License}:

\begin{center}
Copyright (c) 2013 The Board of Trustees of the University of Illinois.
All rights reserved.
\par\end{center}

\begin{center}
Developed by: Department of Civil and Environmental Engineering University
of Illinois at Urbana-Champaign \url{https://github.com/RobinGuers/On-Informational-Nudging}
\par\end{center}

Permission is hereby granted, free of charge, to any person obtaining
a copy of this software and associated documentation files (the \textquotedbl{}Software\textquotedbl{}),
to deal with the Software without restriction, including without limitation
the rights to use, copy, modify, merge, publish, distribute, sublicense,
and/or sell copies of the Software, and to permit persons to whom
the Software is furnished to do so, subject to the following conditions:
Redistributions of source code must retain the above copyright notice,
this list of conditions and the following disclaimers. Redistributions
in binary form must reproduce the above copyright notice, this list
of conditions and the following disclaimers in the documentation and/or
other materials provided with the distribution. Neither the names
of the Department of Civil and Environmental Engineering, the University
of Illinois at Urbana-Champaign, nor the names of its contributors
may be used to endorse or promote products derived from this Software
without specific prior written permission.

THE SOFTWARE IS PROVIDED \textquotedbl{}AS IS\textquotedbl{}, WITHOUT
WARRANTY OF ANY KIND, EXPRESS OR IMPLIED, INCLUDING BUT NOT LIMITED
TO THE WARRANTIES OF MERCHANTABILITY, FITNESS FOR A PARTICULAR PURPOSE
AND NONINFRINGEMENT. IN NO EVENT SHALL THE CONTRIBUTORS OR COPYRIGHT
HOLDERS BE LIABLE FOR ANY CLAIM, DAMAGES OR OTHER LIABILITY, WHETHER
IN AN ACTION OF CONTRACT, TORT OR OTHERWISE, ARISING FROM, OUT OF
OR IN CONNECTION WITH THE SOFTWARE OR THE USE OR OTHER DEALINGS WITH
THE SOFTWARE.


\section{Publishing results using this software}

We kindly ask any future publications using this software include
a reference to the following publication:

R. Guers, C. Langbort and D. Work, ``On informational nudging and control of payoff-based learning'' \emph{accepted
for publication in} \emph{4th IFAC Workshop on Distributed Estimation and Control in Networked Systems}, 2013.

\section{Structure of the code}
Below we give a description of scripts included in the software.
\begin{itemize}
\item \begin{verbatim}Plotl2R21Diagram.m\end{verbatim}
implement the drawing of the stability region presented in subsection 3-2. It uses the functions
\begin{itemize}
\item \begin{verbatim}R21Critical1.m\end{verbatim} \item \begin{verbatim}R21Critical1.m\end{verbatim}
which draw the upper and lower boundaries of the unstable region
\end{itemize}



\end{itemize}
\section{Running the code}
The provided m-files can be used to reproduce the results presented in the publication.
\begin{enumerate}
\item Create Figure 1 presented in subsection 3.2 by running the script\begin{verbatim}Plotl2R21Diagram.m\end{verbatim} where the following notation are .
In this code, $\beta$ the logit parameter, StartingPoint end EndingPoint of $X_{21}$ (the range over which the isoprobability are drawn) and NumPoint (the precision) can be modified.
\end{enumerate}
The default value are $\beta=1$, StartingPoint=-20, EndingPoint=20 and NumPoint=100.



\end{document}
